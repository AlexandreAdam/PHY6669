\documentclass{article}
\usepackage[a4paper, margin=2cm]{geometry}

\usepackage{amsmath}
\usepackage{amssymb}
\usepackage{mathtools}
\usepackage{amstext}
\usepackage{amsthm}
\usepackage{fancyhdr}
\usepackage{siunitx}
\usepackage{physics}

\usepackage{hyperref}


\usepackage{graphicx}
\usepackage{float}
\graphicspath{{figures/}} %Setting the graphicspath
\usepackage{float}
\usepackage{caption}
\usepackage{subcaption}

% To work with inkfigures
\usepackage{import}
\usepackage{pdfpages}
\usepackage{transparent}
\usepackage{xcolor}

\newcommand{\incfig}[2][1]{%
    \def\svgwidth{#1\columnwidth}
    \import{./figures/}{#2.pdf_tex}
}

\pdfsuppresswarningpagegroup=1

%\graphicspath{{figures/}}

\pagestyle{fancy}
\rhead{Alexandre Adam}
\lhead{}
\chead{}
\rfoot{\today}
\cfoot{\thepage}

\newcommand{\angstrom}{\textup{\AA}}
\numberwithin{equation}{section}
\renewcommand\thesubsection{\alph{subsection})}
\renewcommand\thesubsubsection{\Roman{subsubsection}}
\newcommand{\s}{\hspace{0.1cm}}

\newcommand{\pyoutput}[2]{#2}

% Astronomy
\DeclareSIUnit\parsec{pc}
\DeclareSIUnit\lightyear{ly}

\begin{document}
\section{}
La valeur de $a_{\text{rec}}$ (et particulièrement de $z_{\text{rec}}$) 
est très sensible à $X_e$. En principe, on doit résoudre l'équation de 
Boltzmann pour le processus de recombinaison pour déterminer sa valeur. 
Une approximation qui simplifie légèrement notre tâche est d'utiliser 
l'équation de Saha (qui cesse d'être valide très peu de temps 
avant le moment qui nous intéresse). 
\[
		\frac{n_e^{(0)} n_p^{(0)}}{n_H^{(0)}} 
		= \left( \frac{m_e T}{2 \pi} \right)^{3/2}
		\exp \left\{ -\frac{\text{Ry}}{T} \right\} 
\]
Selon notre approximation,
\[
		\frac{n_e^{(0)} n_p^{(0)}}{n_H^{(0)}}
		= \frac{X_en_p}{1 - X_e}
\]
Pour éliminer $n_p$ de l'équation, on utilise $\eta_b \equiv n_b / n_\gamma$:
\[
		\implies n_p = \eta_b X_e n_\gamma = \eta_b X_e\frac{4 \zeta(3)}{\pi^2}T^3
\]
On doit donc résoudre
\[
		\frac{1 - X_e}{X_e^2} = \frac{4 \zeta(3)}{\pi^2}\eta_b \left( \frac{m_e}{2 \pi T} \right)^{-3/2}
		\exp \left\{ \frac{\text{Ry}}{T} \right\} \equiv S(\eta_b, T)
\]
Qui a comme solution physique
\[
		X_e = \frac{-1 + \sqrt{1 + 4S}}{2S}
\]
Sachant que la température suit la loi
\[
		T = T_{\text{CMB}}a^{-1}
\]
au moment de la recombinaison, on peut donc chercher une solution 
à l'équation transcendentale 
\[
		a_{\text{rec}} = 0.6840\, (\Omega_b h^2)
\]
\section{}
Pour résoudre, on doit aussi déterminer $X_e$ en terme du facteur 
d'échelle. Pour ce faire, on doit résoudre l'équation de Boltzmann 
de la réaction de recombinaison $e^{-} + p \longleftrightarrow \gamma + H$. 
Ici, on assume que $n_\gamma = n_\gamma^{(0)}$:
\begin{equation}\label{eq:Boltzmannrecombinaison} 
		a^{-3} \frac{d(n_e a^3)}{dt} = n_e^{(0)}n_p^{(0)} \expval{\sigma v}
		\left\{ \frac{n_H}{n_{H}^{(0)}}  - \frac{n_e^2}{n_e^{(0)} n_p^{(0)}}\right\}
\end{equation} 
On peut simplifier cette expression avec l'équation de Saha 
pour les distributions à l'équilibres:
\[
		\frac{n_e^{(0)} n_p^{(0)}}{n_H^{(0)}} 
		= \left( \frac{m_e T}{2 \pi} \right)^{3/2}
		\exp \left\{ -\frac{\text{Ry}}{T} \right\} 
\]
où $\text{Ry} = 2 \pi \hbar c R_{\infty } = m_e + m_p - m_h$ 
est la constante de Rydberg. Avec $n_h = n_b(1 - X_e)$ on 
obtient
\[
		a^{-3} \frac{d(n_e a^3)}{dt} =  (1 - X_e)n_b\beta  - X_e^2n_b^2\alpha^{(2)}
\]
où $\beta$ est le taux d'ionisation (on réintroduit les unités physiques)
\[
		\beta \equiv 
		\alpha^{(2)} c\left( \frac{m_e k_B T}{2 \pi \hbar^{2}} \right)^{3/2}
		\exp \left\{ -\frac{\text{Ry}}{k_B T} \right\}
\]
et $\alpha^{(2)} = \expval{\sigma v}$ est la section efficace thermalisée 
excluant 
le niveau fondamental $n = 1$ puisque le photon de recombinaison
de cet état état va ioniser immédiatement un atome $H$ voisin. 
On omet le détail de ce calcul et on se réfère à l'équation (3.42)
de Dodelson pour une approximation de cette section efficace
\[
		\alpha^{(2)} = 9.78 \frac{\hbar^2\alpha^{2}}{m_e^2c^2} 
		\left( \frac{\text{Ry}}{k_BT}  \right)^{1/2}
		\ln \left( \frac{\text{Ry}}{k_BT} \right)
\]
Sachant qu'au moment de la recombinaison, la température de la 
matière est approximativement la même que la température des 
photons à cause des processus de diffusion, on a
\[
		T = T_{\text{CMB}}a^{-1} = 2.725\,48\, a^{-1}\,\,\text{K}
\]
On a donc des expressions en terme du facteur d'échelle pour 
la plupart des quantités d'intérêt. On revient à l'équation de 
Boltzmann pour changer la variable d'intégration et simplifier le 
côté gauche. En effet, on réalise que $n_e a^3 \simeq X_e n_b a^3$. 
D'où
\[
		\frac{dX_e}{dt} =  (1 - X_e(a))\beta(a)  - X_e^2(a)n_b\alpha^{(2)}(a)
\]
On utilise la seconde équation de Friedmann pour une expression 
de $dt$ dans le régime matière-radiation qui nous intéresse:
\[
		dt = \frac{da}{H_0 \sqrt{\Omega_0 a^{-2} + \Omega_m a^{-1}}}
\]
d'où l'équation différentielle
\[
		\frac{dX_e}{da} = (100\, \text{km}\,\text{s}^{-1}\, \text{Mpc}^{-1})^{-1}
		\frac{ (1 - X_e(a))\beta(a)  - X_e^2(a)n_b\alpha^{(2)}(a)}{
		h\sqrt{\Omega_r a^{-2} + \Omega_m a^{-1}}}
\]


\end{document}

